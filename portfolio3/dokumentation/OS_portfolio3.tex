% !TeX spellcheck = de_DE
\documentclass[12pt]{article}
\usepackage[utf8]{inputenc}
\usepackage{geometry}
\usepackage{svg}
\usepackage{float}
\usepackage{caption}
\usepackage{amsmath,amsthm,amsfonts,amssymb,amscd}
\usepackage{fancyhdr}
\usepackage{titlesec}
\usepackage{hyperref}
\usepackage{listings}
\usepackage[skip=3pt]{parskip}
\usepackage[ngerman]{babel}
\pagestyle{empty}
\titleformat*{\section}{\large\bfseries}
\titleformat*{\subsection}{\bfseries}

%
\geometry{
	a4paper,
	total={170mm,240mm},
	left=20mm,
	top=30mm,
}

\date{}
%Bitte ausfüllen
\newcommand\course{Betriebssysteme}
\newcommand\hwnumber{\large Portfolio 3}
\newcommand\Name{Fabian Sponholz}
\newcommand\Neptun{1561546}

%Matheinheiten
\newcommand\m{\:\textrm{m}}
\newcommand\M{\:\Big[\textrm{m}\Big]}
\newcommand\mm{\:\textrm{mm}}
\newcommand\MM{\:\Big[\textrm{mm}\Big]}
\newcommand\un{\underline}
\newcommand\s{\:\textrm{s}}
\newcommand\bS{\:\Big[\textrm{S}\Big]}
\newcommand\ms{\:\frac{\textrm{m}}{\textrm{s}}}
\newcommand\MS{\:\Big[\frac{\textrm{m}}{\textrm{s}}\Big]}
\newcommand\mss{\:\frac{\textrm{m}}{\textrm{s}^2}}
\newcommand\MSS{\:\Big[\frac{\textrm{m}}{\textrm{s}^2}\Big]}

%Trennlinie
\newcommand\separator{\rule{\linewidth}{0.5pt}}

%Bitte nicht einstellen
\renewcommand{\figurename}{Abbildung}
\renewcommand{\tablename}{Tabelle}
\pagestyle{fancyplain}
\headheight 35pt
\lhead{\Name\\\Neptun}
\chead{\textbf{ \hwnumber}}
\rhead{\course \\ \today}
\lfoot{}
\cfoot{}
\rfoot{\small\thepage}
\headsep 1.5em

\lstset{
	basicstyle=\ttfamily\small,
	columns=fullflexible,
	frame=single,
	frameround=tttt,
	rulecolor=\color{gray},
}
% http://www.bollchen.de/blog/2011/04/good-looking-line-breaks-with-the-listings-package/
\lstset{
	prebreak=\raisebox{0ex}[0ex][0ex]{\ensuremath{\hookleftarrow}},
	postbreak=\raisebox{0ex}[0ex][0ex]{\ensuremath{\hookrightarrow\space}},
	breaklines=true,
	breakatwhitespace=true,
	numbers=left,
	numberstyle=\scriptsize,
}
\lstset{
	backgroundcolor=\color{white},
	extendedchars=true,
	basicstyle=\footnotesize\ttfamily,
	showstringspaces=false,
	showspaces=false,
	numbers=left,
	numberstyle=\footnotesize,
	numbersep=9pt,
	tabsize=2,
	breaklines=true,
	showtabs=false,
	captionpos=b
}
\lstset{
	keywordstyle=\color{blue}\bfseries,
	ndkeywordstyle=\color{darkgray}\bfseries,
	identifierstyle=\color{black},
	commentstyle=\color{purple}\ttfamily,
	stringstyle=\color{red}\ttfamily
}

\begin{document}
	
\section*{Erste Überlegungen zur Umsetzung}
Um die drei geforderten Funktionalitäten umzusetzen, denke ich, dass man diese in drei Methoden gießen sollte:
Eine zum Lesen, eine zum Schreiben und eine zum Löschen von Dateien.

\subsection*{Lesen von Dateien}
Um Dateien konsistent zu lesen, ohne Dateisperren zu verwenden, muss vor dem Lesen ein ZFS Snapshot erstellt werden.
Der Inhalt der Datei wird dann aus dem Snapshot gelesen, anstelle des eigentlichen Dateisystems.
So wird sichergestellt, dass parallele Schreibvorgänge auf der selben Datei keine Inkonsistenzen hervorrufen:
Sollte ein paralleler Schreibzugriff stattfinden, wird der Snapshot aufgrund des Copy-On-Write-Prinzips immer die unveränderte Version vor dem Schreibvorgang beinhalten.

\subsection*{Löschen von Dateien}
Das Löschen von Dateien durch die \texttt{unlink()}-Funktion oder den \texttt{rm}-Befehl sind auf Linux-Systemen atomar.
Daher ergibt es keinen Sinn, vor dem Löschen einen Snapshot zu erstellen.
Auftretende Konflikte müssen aber bei der Read-Write-Transaktion erkannt werden.

\subsection*{Lesen und Schreiben von Dateien}
Hier wird zunächst eine Transaktion geöffnet, indem ein Snapshot erstellt wird. 
Dann wird dem Nutzer der Dateiinhalt aus dem Snapshot zur Verfügung gestellt, wie beim normalen Lesen.
Nun kann der Nutzer die Datei bearbeiten, den neuen Dateiinhalt schreiben und die Transaktion "committen".
Nun wird geprüft, ob der Nutzer die Datei überhaupt verändert hat.
Falls nicht, wird einfach nichts getan und der Snapshot wieder entfernt.
Falls doch, wird noch auf inkonsistenzen geprüft: Eine Prüfsumme wird auf der Datei aus dem Snapshot erstellt, sowie aus der Datei vom Live-System.
Sollte sich die Datei geändert haben, wird ein Rollback durchgeführt.
Falls nicht, wird die Datei geschrieben und der Snapshot gelöscht.

\section*{Umsetzung der Transaction-Library}
Meine Transaktions-Bibliothek besteht aus zwei Klassen. 
Eine Klasse \texttt{ZFSUtil}, die als Interface zwischen Java und ZFS fungiert, indem sie Methoden zum Verwalten von Dateien und Snapshots bereitstellt und diese auf ZFS-Befehle abbildet.

Zweitens eine Klasse \texttt{ZFSTransaction}, die die Anforderungen der Aufgabe mithilfe der ZFSUtils-Klasse umsetzt.
Sie ermöglicht es, eine Transaktion zu öffnen, innerhalb der Transaktion Dateien zu lesen, zu bearbeiten und zu löschen, sowie die Transaktion am Ende zu committen.
Erst beim Commit werden tatsächlich Änderungen am Live-Dateisystem vorgenommen. 
Sollten Konflikte erkannt werden, wird ein Rollback durchgeführt und die Transaktion somit rückgängig gemacht.

\subsection*{ZFSUtil: Implementierungsdetails}
Zunächst einmal hat die Klasse ZFSUtils drei Konfigurations-Variablen: \texttt{ZFS\_MOUNTPOINT},\linebreak \texttt{ZFS\_SNAPSHOT\_DIRECTORY} und \texttt{ZFS\_FILESYSTEM}.
Über diese kann das zu verwaltende ZFS-Dateisystem spezifiziert werden.
Standardmäßig wird ein Dateisystem \texttt{mypool/myfs} verwendet, mit dem Standard-Mountpoint \texttt{/mypool/myfs} und dem Standart-Snapshot-Directory \linebreak \texttt{/mypool/myfs/.zfs/snapshot}

Die grundlegende Funktionalität liegt in den Methoden \texttt{createSnapshot()}, \texttt{deleteSnapshot()} und \texttt{rollbackSnapshot()}.
Diese verwenden die \texttt{Runtime.exec()}-Methode, um ZFS-Kommandos auf dem Hostsystem auszuführen und so ZFS-Snapshots zu verwalten.
Mithilfe der \texttt{Process.waitFor()}-Methode wird jeweils gewartet, bis der erzeugte Prozess terminiert, um so eine Synchronisierung mit dem Java-Programm zu erreichen.

Zusätzlich stehen noch Methoden zum Lesen von Dateien aus dem Live-Filesystem und aus Snapshots zur Verfügung, sowie Methoden zum Schreiben und Löschen von Dateien im Live-Filesystem.
Hier kommt die Java IO- bzw NIO-API zum Einsatz.

Zuletzt gibt es zum späteren Verifizieren der Konsistenz noch Methoden zum holen des File-Objekts einer Datei im Live-Filesystem bzw. in einem Snapshot, um daraus später den Zeitstempel zu extrahieren.

\subsection*{ZFSTransaction: Implementierungsdetails}
Ein Objekt der Klasse \texttt{ZFSTransaction} hat zwei Attribute: Eine UUID, die die Transaktion eindeutig identifiziert und eine Map, in der Datei-Änderungen innerhalb der Transaktion verfolgt werden.

Mithilfe der statischen Methode \texttt{ZFSTransaction.open()} wird eine neue Transaktion geöffnet, indem ein neues ZFSTransaction-Objekt initialisiert und ihm eine zufällige UUID zugewiesen wird.
Dann wird ein ZFS-Snapshot erstellt, wobei der Name des Snapshots der UUID der Transaktion entspricht.
So kann die Transaktion später durch einen Rollback auf den Snapshot zurückgesetzt werden.

Nun hält das ZFSTransaction-Objekt einige Methoden bereit, um mit den Dateien zu interagieren.
Zunächst gibt es die private Methode \texttt{openFile()}, die den Inhalt einer Datei aus dem Snapshot der Transaktion in die \texttt{files}-Map lädt.
Ist die Datei nicht vorhanden, wird ein leerer String in die Map geschrieben, um das erstellen einer neuen, leeren Datei zu simulieren.

Folgende Methoden stehen zur direkten Interaktion bereit:

\begin{itemize}
	\item Die \texttt{readFile()}-Methode liest den Inhalt einer Datei aus der \texttt{files}-Map aus und gibt diese zurück.
	Falls die Datei nicht enthalten ist, wird sie mithilfe der \texttt{openFile()}-Methode geladen.
	
	\item Die \texttt{writeFile()}-Methode schreibt den Inhalt für eine Datei in die \texttt{files}-Map.
	So wird der Inhalt bei Abschluss der Transaktion in die Datei geschrieben.
	
	\item Die \texttt{deleteFile()}-Methode setzt den Inhalt einer Datei in der \texttt{files}-Map auf \texttt{null}.
	So wird die Datei bei Abschluss der Transaktion gelöscht.
	
	\item Die \texttt{close()}-Methode schließt eine Transaktion ab, indem der entsprechende Snapshot im ZFS gelöscht wird.
	Zudem wird die \texttt{files}-Map geleert und die UUID auf \texttt{null} gesetzt, wodurch das ZFSTransaction-Objekt nicht weiter verwendet werden kann.
	
	\item Die \texttt{rollback()}-Methode führt einen Rollback auf den Snapshot vor der Transaktion durch. 
	Dadurch werden alle Veränderungen seit Transaktionsbeginn verworfen.
	Zuletzt wird die \texttt{close()}-Methode aufgerufen.
	
	\item Die \texttt{commit()}-Methode wendet die Änderungen, die in der \texttt{files}-Map gespeichert wurden, an.
	Für jede Datei wird zunächst die Konsistenz geprüft, indem der Zeitstempel der Datei aus dem Snapshot mit der Datei aus dem Live-System verglichen wird.
	Falls die Zeitstempel übereinstimmen, wird der neue Dateiinhalt geschrieben.
	Wird an irgendeiner Stelle eine Inkonsistenz (Abweichender Zeitstempel) festgestellt, so wird der Schreibvorgang abgebrochen und ein Rollback durchgeführt.
	Zuletzt wird die \texttt{close()}-Methode aufgerufen und die Transaktion somit geschlossen.
\end{itemize}



\end{document}