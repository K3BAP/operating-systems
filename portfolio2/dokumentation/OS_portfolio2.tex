\documentclass[12pt]{article}
\usepackage[utf8]{inputenc}
\usepackage{geometry}
\usepackage{svg}
\usepackage{float}
\usepackage{caption}
\usepackage{amsmath,amsthm,amsfonts,amssymb,amscd}
\usepackage{fancyhdr}
\usepackage{titlesec}
\usepackage{hyperref}
\usepackage{listings}
\usepackage[skip=3pt]{parskip}
\usepackage[ngerman]{babel}
\pagestyle{empty}
\titleformat*{\section}{\large\bfseries}
\titleformat*{\subsection}{\bfseries}

%
\geometry{
	a4paper,
	total={170mm,240mm},
	left=20mm,
	top=30mm,
}

\date{}
%Bitte ausfüllen
\newcommand\course{Betriebssysteme}
\newcommand\hwnumber{\large Portfolio 2}
\newcommand\Name{Fabian Sponholz}
\newcommand\Neptun{1561546}

%Matheinheiten
\newcommand\m{\:\textrm{m}}
\newcommand\M{\:\Big[\textrm{m}\Big]}
\newcommand\mm{\:\textrm{mm}}
\newcommand\MM{\:\Big[\textrm{mm}\Big]}
\newcommand\un{\underline}
\newcommand\s{\:\textrm{s}}
\newcommand\bS{\:\Big[\textrm{S}\Big]}
\newcommand\ms{\:\frac{\textrm{m}}{\textrm{s}}}
\newcommand\MS{\:\Big[\frac{\textrm{m}}{\textrm{s}}\Big]}
\newcommand\mss{\:\frac{\textrm{m}}{\textrm{s}^2}}
\newcommand\MSS{\:\Big[\frac{\textrm{m}}{\textrm{s}^2}\Big]}

%Trennlinie
\newcommand\separator{\rule{\linewidth}{0.5pt}}

%Bitte nicht einstellen
\renewcommand{\figurename}{Abbildung}
\renewcommand{\tablename}{Tabelle}
\pagestyle{fancyplain}
\headheight 35pt
\lhead{\Name\\\Neptun}
\chead{\textbf{ \hwnumber}}
\rhead{\course \\ \today}
\lfoot{}
\cfoot{}
\rfoot{\small\thepage}
\headsep 1.5em

\begin{document}
	
\section*{Umgebung der Experimente}
Folgende Tabellen beschreiben das System, auf dem die Tests durchgeführt wurden.
\subsection*{Hardware-Spezifikation}
\begin{table}[h]
	\centering
	\begin{tabular}{|l|l|}
		\hline
		\textbf{Merkmal} & \textbf{Spezifikation} \\
		\hline
		CPU-Bezeichnung & AMD Ryzen 5 4500U with Radeon Graphics\\
		CPU-Architektur & x86 (AMD Renoir) \\
		CPU-Fertigungsverfahren & 7 nm (TSMC) \\
		\hline
		Anzahl Kerne / Threads & 6 / 6 \\
		Basistaktfrequenz & 2.3 GHz \\
		Maximale Boost-Taktfrequenz & 4.0 GHz \\
		\hline
		L1-Cache & 384 KB \\
		L2-Cache & 3 MB \\
		L3-Cache & 8 MB \\
		\hline
		TDP (Thermal Design Power) & 15 Watt \\
		Maximale Temperatur & 105 °C \\
		\hline
		Arbeitsspeicher & 8GB DDR4-3200 \\
		\hline
		Erscheinungsdatum & 07.01.2020 \\
		\hline
	\end{tabular}
\end{table}

\subsection*{Software-Umgebung}
\begin{table}[h]
	\centering
	\begin{tabular}{|l|l|}
		\hline
		\textbf{Merkmal} & \textbf{Spezifikation} \\
		\hline
		Betriebssystem & Arch Linux\\
		Kernel-Version & Linux 6.12.8-arch1-1\\
		\hline
		Java-Version & Java 21\\
		Java-Implementierung & java-21-openjdk\\
		\hline
	\end{tabular}
\end{table}

\section{Aufgabe 1 - Latenz bei Kommunikation mit Spinlock}
\subsection*{Vorgehen zur Latenzmessung}
Bei der Kommunikation über Spinlocks wartet ein Thread auf die Freigabe einer Ressource, indem er fortwährend (z.B. in einer \texttt{while}-Schleife) überprüft, ob die Ressource frei ist. 
Um die Latenz zu messen, habe ich neben dem Main-Thread einen \texttt{Reader}-Thread erstellt, der auf die Freigabe einer Ressource (Boolean, der auf \texttt{true} gesetzt wird) wartet, und den Wert dann wieder auf \texttt{false} setzt. 
Nachdem der Main-Thread den Wert auf \texttt{true} gesetzt hat, wartet er wiederum, bis der Wert wieder auf \texttt{false} gesetzt wird.

Um Race Conditions beim Abfragen der Werte aus den While-Schleifen zu vermeiden, wird der Zugriff mithilfe von \texttt{synchronized} Setter- und Getter-Methoden geregelt.
Jeder Thread trägt immer, wenn er eine Nachricht vom anderen Thread erhält, einen Zeitstempel in Nanosekunden in eine ausreichend große Array-Liste ein, woraus später die Latenz berechnet wird.
Nun folgen Auszüge aus dem Source Code, die dies zeigen.

\begin{lstlisting}[language=java,caption={Latenzmessung im Main Thread}]
for (int i = 0; i < recursions; i++) {
	// set lock to true -> send message
	setLock(true);;
	
	// wait for the return message
	while (getLock()) {
		// Do nothing (spinlock)
	}
	
	// add current time to list
	messageTimes.add(System.nanoTime());
}
\end{lstlisting}

\begin{lstlisting}[language=java,caption={Latenzmessung im Reader Thread}]
while (!isInterrupted()) {
	while (!isInterrupted() && experiment.getLock() == false) {
		// do nothing (spinlock)
	}
	
	// Add the current time to the list
	experiment.getMessageTimes().add(System.nanoTime());
	
	if (isInterrupted()) break;
	
	// Reset the lock (main thread is waiting)
	experiment.setLock(false);
}
\end{lstlisting}


\subsection*{Versuchsaufbau / Implementierung}
Im Ursprünglichen Versuchsaufbau habe ich für jeden Versuchsdurchlauf den \texttt{Reader}-Thread und das \texttt{Experiment}-Objekt neu erstellt und jeweils von jedem Experiment die minimale Latenz gespeichert.
Das Ergebnis des ersten Durchlaufs ist in Abbildung \ref{img:spinlock_first} dargestellt.

Wie man sieht liegt, anders als zu erwarten wäre, keine Normalverteilung der Latenzen vor.
Nach reiflicher Überlegung hatte ich den Verdacht, dass ich durch die Instantiierung einer \texttt{Experiment}- und \texttt{Reader}-Objekts bei jedem Versuch eine große Menge ungenutzter Objekte erzeuge und dadurch ggf. der \emph{Garbage Collector} die Performance zeitweise mindert.

Nachdem ich die Implementierung so angepasst habe, dass die bestehenden Objekte in jedem Experiment wiederverwendet werden können, ergab sich tatsächlich eine Normalverteilung der Minima, sodass das 95\%-Konfidenzintervall einfach berechnet werden konnte. Mehr dazu in Abschnitt \ref{sect:results}.

\begin{figure}[H]
	\centering
	\includegraphics[width=0.75\textwidth]{./img/spinlock_first_try}
	\caption{Verteilung der Minima im ersten Experiment}
	\label{img:spinlock_first}
\end{figure}


\section{Ergebnisse}
\label{sect:results}

\subsection*{Aufgabe 1 - Spinlock}
Es wurden 1000 Durchläufe des Experiments durchgeführt mit jeweils 50000 Wiederholungen der Messung, wobei jeweils zwei Latenzen gemessen wurden (Hin- und Rückweg).
Insgesamt wurden pro Experiment also 100000 Latenzen gemessen und jeweils das Minimum gespeichert.
Hier die Ergebnisse:
\begin{figure}[H]
	\centering
	\includegraphics[width=0.75\textwidth]{./img/spinlock_second}
	\caption{Verteilung der Minima mit Konfidenzintervall (blau)}
	\label{img:spinlock_second}
\end{figure}

\begin{itemize}
	\item Durchschnittliche minimale Latenz: $195,31 ns$
	\item 95\%-Konfidenzintervall: $194,39 ns - 196,22 ns$
\end{itemize}


\end{document}